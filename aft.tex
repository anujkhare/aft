\documentclass[12pt,a4paper]{report}

\usepackage{amsthm,amssymb,mathrsfs,setspace}%amsmath, latexsym,footmisc

% \usepackage{pstcol}
% \usepackage{play}
\usepackage[document]{ragged2e}
\usepackage{epsfig}
%\usepackage[grey,times]{quotchap}
\usepackage[nottoc]{tocbibind}
\usepackage{amsmath}
%% CREATE BOOKMARKS IN THE PDF!
\usepackage[bookmarks]{hyperref}
%\usepackage[hidelinks]{hyperref}
\usepackage{algorithm2e}

\renewcommand{\chaptermark}[1]{\markboth{#1}{}}
\renewcommand{\sectionmark}[1]{\markright{\thesection\ #1}}
%

\input xy
\xyoption{all}

\begin{document}
\begin{center} \section*{Regularized AFT models} Anuj Khare \end{center}
%\noindent\makebox[\linewidth]{\rule{\paperwidth}{0.4pt}}
\vspace{8mm}
A standard AFT model is defined as follows:


\begin{equation}
	\log (Y_i) = \beta_0  + x_i^T \beta + \sigma \epsilon_i
\end{equation}

Where $x_i$ are the covariates, $Y_i$ is the observed time (output). $\epsilon_i ~ F$, where $F$ is the distribution function.

Consider $F$ to be the logistic cdf, given as:
\begin{equation} \label{cdf}
F(x) = \frac{1}{1 + e^{-x}}
\end{equation}

The probability density function is given as:
\begin{equation} \label{pdf}
f(x) = \frac{e^{-x}}{(1 + e^{-x})^2}
\end{equation}

Survival function can be calulated from the above:
\begin{equation} \label{survival}
1 - F(x) = \frac{1}{1 + e^{x}}
\end{equation}

Here on, we assume that $\sigma$ is constant for each observation $i$, and is ignored, and that the covariate matrix is appended with a column of ones.
%Define $\eta = X'\beta$ as the vector of linear predictors.
Hence,

\begin{equation}
	\epsilon_i = \frac{\log (y_i) - (x_i^T \beta)} {\sigma}
\end{equation}


For interval regression with censored data, we are given time intervals $\{\underline t_i, \overline t_i\}$ and covariates $x_i$ for $i=1:n$, where $\underline t_i$ may be $-inf$
(left censoring) and $\overline t_i$ may be $inf$ (right censoring).

%%%%%%%%%%%%%%%%%%%%%%%%%%%%%%%%%%%%%%%%%%%% LIKELIHOOD %%%%%%%%%%%%%%%%%%%%%%%%%%%%%%%%%%
\subsection*{Likelihood}
For calculating likelihood, in the observations with no censoring, the pdf is used, and in censored observations, the cdf is used.
The likelihood is given as:

\begin{equation}
L(\beta) = \prod_{i=1}^n \zeta(\beta, \underline t_i, \overline t_i)
\end{equation}

where,
\begin{equation} \label{zeta}
\zeta_i = \zeta(\beta, \underline t_i, \overline t_i) = \begin{cases}
													   \zeta_{i,1} = F(R_i) & \mbox{if: } -\infty \textbf{ = } \underline t_i \mbox{ , } \overline t_i<\infty \\
													   \zeta_{i,2} = 1- F(L_i) & \mbox{if: } -\infty < \underline t_i \mbox{ , }\overline t_i\textbf{ = }\infty \\
													   \zeta_{i,3} = F(R_i) - F(L_i) & \mbox{if: } -\infty < \underline t_i \ne \overline t_i<\infty \\
													   \zeta_{i,4} = d(R_i) & \mbox{if: } -\infty < \underline t_i \textbf{ = } \overline t_i<\infty \\
		\end{cases}
\end{equation}

where, $R_i = \log (\overline t_i) - \eta_i$, and $L_i = \log (\underline t_i) - \eta_i$. $\eta_i = x_i^T \beta$ is the vector of linear predictors.
Clearly, for left censored observations, $\underline t_i = -inf$, hence we can substitute $L_i = 0$ in $\zeta_3$ to get $\zeta_2$. 
Simlarly, for right censored observations, $\overline t_i = inf$, hence we can substitute $R_i = inf$ in $\zeta_3$ to get $\zeta_1$. \\

\vspace{8mm}
Hence the log likelihood is given as:

\begin{equation} \label{lik0}
l(\beta) = \sum_{i=1}^n \log \zeta_i = \sum_{i=1}^n \gamma_i %(\beta, \underline t_i, \overline t_i)
\end{equation}

where, $\gamma_i = \log \zeta_i$.

%%%%%%%%%%%%%%%%%%%%%%%%%%%%%%%%%%%%%%%%%%%% SCORE %%%%%%%%%%%%%%%%%%%%%%%%%%%%%%%%%%
\subsection*{Score}
For now, we consider that $\underline t_i = \overline t_i$ is not possible. Hence we are left with the types of censoring. \\
We consider the parital derivative  of $\gamma_{i,3}$ wrt to the parameters $\beta$. We can simply obtain derivatives for
$\gamma_{i,1}$ and $\gamma_{i,2}$ by setting $L_i = 0$ and $R_i = inf$ respectively.\\

We use the following:
\begin{equation} \label{dF}
\frac{\partial F(R_i)}{\partial \beta_j} =  -x_{ij} F(R_i)[1-F(R_i)]
\end{equation}

\begin{equation}
\frac{\partial \gamma_{i,3}}{\partial \beta_j} = - x_{ij} \frac{ [F(R_i) (1-F(R_i)) - F(L_i)(1-F(L_i))] }{F(R_i) - F(L_i)} = x_{ij} [F(L_i) + F(R_i) - 1]
\end{equation}

Define $\mu_i = (F(L_i) + F(R_i) - 1)$, $\mu = [\mu_1, ... \mu_n]^T$. Then, partial derivative of log-likelihood is given as:

\begin{equation}
\frac{\partial l(\beta)}{\partial \beta_j} = \sum_{i=1}^n x_{ij} \mu_i
\end{equation}

Hence, the score (gradient of llikelihood) is given as:

\begin{equation}
g = \nabla_{\beta} l(\beta) = X^T \mu = \sum_{i=1}^n \mu_i \overline x_i
\end{equation}

%%%%%%%%%%%%%%%%%%%%%%%%%%%%%%%%%%%%%%%%%%%% HESSIAN %%%%%%%%%%%%%%%%%%%%%%%%%%%%%%%%%%
\subsection*{Hessian}
Hessian is the second derivative of loglikelihood:

\begin{equation}
H = \frac{ \partial {g(\beta)}^T}{\partial \beta} = \sum_{i=1}^n (\nabla_{\beta} \mu_i) \overline x_i^T
\end{equation}

Using \ref{dF}, we have:

\begin{equation}
\frac{ \partial \mu_i}{\partial \beta_j} = -x_{ij} [F(R_i) (1-F(R_i)) +  F(L_i) (1-F(L_i))]
\end{equation}

\begin{equation}
\nabla_{\beta} \mu_i = -\overline x_i [F(R_i) (1-F(R_i)) + F(L_i) (1-F(L_i))]
\end{equation}

Hence, the hessian is given as:
\begin{equation}
H = \sum_{i=1}^n -w_i \overline x_i \overline x_i^T
\end{equation}

where, $w_i$ is given as:
\begin{equation} \label{w_i}
\begin{split}
w_i & = [F(R_i) (1-F(R_i)) + F(L_i) (1-F(L_i))] \\
	& = - \mu_i [F(R_i) - F(L_i)]
\end{split}
\end{equation}

Define $W=diag(w_1, ... w_n)$.
\begin{equation}
H = - X^T W X
\end{equation}

Clearly, the hessian is negative definite. Thus, the loglikelihood is strictly convex, and a unique global
maximum exists.

%%%%%%%%%%%%%%%%%%%%%%%%%%%%%%%%%%%%%%%%%%%% ELASTIC  %%%%%%%%%%%%%%%%%%%%%%%%%%%%%%%%%%
\subsection*{IRLS}
We use Newton's algorithm to find MLE for the AFT model, using negative loglikelihood (NLL). The Newton update is as follows:
\begin{equation}
\begin{split}
\beta & = \widetilde{\beta} - H^{-1}\widetilde{g} \\ 
 	  & = \widetilde{\beta} - (X^T \widetilde{W} X)^{-1} X^T \widetilde{\mu} \\
 	  & = (X^T \widetilde{W} X)^{-1} ((X^T \widetilde{W} X) \widetilde{\beta} -  X^T \widetilde{\mu} ) \\
 	  & = (X^T \widetilde{W} X)^{-1} X^T (\widetilde{W} X \widetilde{\beta} -  \widetilde{\mu} ) \\
 	  & = (X^T \widetilde{W} X)^{-1} X^T (\widetilde{W} X \widetilde{\beta} -  \widetilde{\mu} )
\end{split}
\end{equation}

where, define the working response $\widetilde{z} = X \widetilde{\beta} - \widetilde{W}^{-1} \widetilde{\mu}$.
Here, the tilde denotes that the respective values are evaluated using the parameters from the previous step.

Hence, at each step we are solving a weighted least squares problem, which is a minimizer of:

\begin{equation}
\sum_{i=1}^n \widetilde{w_i} (\widetilde{z_i} - \overline x_i^T \beta)^2
\end{equation}

We can rewrite $z$ as:
\begin{equation} \label{z_i}
\begin{split}
\widetilde{ z_i} & = x_i^T \widetilde{ \beta} + \frac{ \widetilde{ \mu_i}} {\widetilde{ w_i}} \\
				 & = x_i^T \widetilde{ \beta} - \frac{ 1}{F(R_i) - F(L_i)}
\end{split}
\end{equation}

This algorithm is the iteratively reweighted least squares (IRLS), since at each iteration we solve a weighted least squares problem.

%%%%%%%%%%%%%%%%%%%%%%%%%%%%%%%%%%%%%%%%%%%% ELASTIC  %%%%%%%%%%%%%%%%%%%%%%%%%%%%%%%%%%
\subsection*{Elastic net penalty and coordinate descent}
We define the elastic net (L1 + L2) penalty as follows:
\begin{equation} \label{objective}
\lambda P_{\alpha}(\beta) = \lambda(\alpha \|\beta\|_1 + 1/2 (1-\alpha) \|\beta\|_2^2)
\end{equation}

Adding the elastic net (L1 + L2) penalty, we get the following penalized weighted least squares objective:
\begin{equation} \label{objective}
M = \sum_{i=1}^n \widetilde{w_i} (\widetilde{z_i} - \overline x_i^T \beta)^2
	 + \lambda P_{\alpha}(\beta)
\end{equation}

The subderivative of the optimization objective is given as:
\begin{equation}
\frac{ \partial M}{\partial \beta_k} = \sum_{i=1}^n \widetilde{w_i} x_{ik} (\widetilde{ z_i} - \overline x_i^T \beta ) + \lambda \alpha \mbox{ sgn}(\beta_k) + \lambda (1-\alpha)\beta_k
\end{equation}

where, sgn$(\beta_k)$ is 1 if $\beta_k > 1$, -1 if $\beta_k<0$ and 0 if $\beta_k = 0$.
%\begin{equation} \label{subderivative}
%\frac{\partial{f}}{\partial{w_j}} = \begin{cases}
%										\{-\frac{\partial{l(\beta)}}{\partial{w_j}} + \lambda_2 w_j -\lambda_1\} & \mbox{if }  w_j<0 \\
%										[-\frac{\partial{l(\beta)}}{\partial{w_j}} -\lambda_1, -\frac{\partial{l(\beta)}}{\partial{w_j}} +\lambda_1] & \mbox{if }  w_j=0 \\
%										\{-\frac{\partial{l(\beta)}}{\partial{w_j}} + \lambda_2 w_j +\lambda_1\} & \mbox{if } w_j>0 \\
%									\end{cases}
%\end{equation}

Using the subderivative, three cases of solutions for $\beta_k$ may be obtained. The solution is given by:

\begin{equation}
\hat \beta_k = \frac{S\left(\sum_{i=1}^n \widetilde{w_i} x_{ik} \left[\widetilde{ z_i} - \sum_{j \ne k} x_{ij} \beta_j \right], \lambda \alpha \right)}
					{\sum_{i=1}^p \widetilde{w_i} x_{ik}^2 + \lambda (1- \alpha)}
\end{equation}

where, S is the soft thresholding operator, and $w_i$ and $z_i$ are given in \ref{w_i} and \ref{z_i} respectively.

The coordinate descent algorithm works by
cycling through each $w_j$ in turn, keeping the others constant, and using the above estimate to calculate the optimal value
$\beta^*_j$. This is repeated until convergence.

%%%%%%%%%%%%%%%%%%%%%%%%%%%%%%%%%%%%%%%%%%%% ALGORITHM  %%%%%%%%%%%%%%%%%%%%%%%%%%%%%%%%%%
\subsection*{Algorithm}
As explained in \cite{coxnet}, the algorithm to be followed for fitting the distribution is:
\begin{enumerate}
\item Initialize $\widetilde{ \beta}$.
\item Compute $\widetilde{ w_i}$ and $\widetilde{ z_i}$.
\item Find $\hat \beta$ by solving the penalized weighted least square problem defined in \ref{objective} using coordinate descent.
\item Set $\widetilde{ \beta} = \hat \beta$
\item Repeat steps 2-4 until convergence of $\hat \beta$
\end{enumerate}

\vspace{8mm}
\textbf{Sources:}
\begin{itemize}
\item Machine Learning: A Probabilistic Perspective - Kevin Murphy
\item Regularization Paths for Cox's Proportional Hazards Model via Coordinate Descent - Simon, Friedman, Hastie, Tibshirani
\item AFT - TD Hocking
\end{itemize}

\end{document}
