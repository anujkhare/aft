\documentclass[12pt,a4paper]{report}

\usepackage{amsthm,amssymb,mathrsfs,setspace}%amsmath, latexsym,footmisc

% \usepackage{pstcol}
% \usepackage{play}
\usepackage[document]{ragged2e}
\usepackage{epsfig}
%\usepackage[grey,times]{quotchap}
\usepackage[nottoc]{tocbibind}
\usepackage{amsmath}
%% CREATE BOOKMARKS IN THE PDF!
\usepackage[bookmarks]{hyperref}
%\usepackage[hidelinks]{hyperref}
\usepackage{algorithm2e}

\renewcommand{\chaptermark}[1]{\markboth{#1}{}}
\renewcommand{\sectionmark}[1]{\markright{\thesection\ #1}}
%

\input xy
\xyoption{all}

\begin{document}
\begin{center} \section*{Regularized AFT models} Anuj Khare \end{center}
%\noindent\makebox[\linewidth]{\rule{\paperwidth}{0.4pt}}
This has been adapted from the Survival package vignette.
\subsection*{Model definition}
A standard AFT model is defined as follows:


\begin{equation}
	\log (T_i) = x_i^T \beta + \sigma \epsilon_i
\end{equation}

Where $x_i$ are the covariates, $Y_i$ is the observed time (output). $\epsilon_i \sim f$, where $f$ is the probability density.

%Consider $F$ to be the logistic cdf, given as:
%\begin{equation} \label{cdf}
%F(x) = \frac{1}{1 + e^{-x}}
%\end{equation}
%
%The probability density function is given as:
%\begin{equation} \label{pdf}
%f(x) = \frac{e^{-x}}{(1 + e^{-x})^2}
%\end{equation}
%
%Survival function can be calulated from the above:
%\begin{equation} \label{survival}
%1 - F(x) = \frac{1}{1 + e^{x}}
%\end{equation}

Here on, we assume that $\sigma$ is fixed, and is ignored.
%Define $\eta = X'\beta$ as the vector of linear predictors.
We let $y_i$ be the transformed data vector obtained by taking log of $T_i$. Hence, we have:

\begin{equation}
	e_i = \frac{y_i - x_i^T \beta} {\sigma} \sim f
\end{equation}


For interval regression with censored data, we are given time intervals $\{\underline t_i, \overline t_i\}$ and covariates $x_i$ for $i=1:n$, where $\underline t_i$ may be $-inf$
(left censoring) and $\overline t_i$ may be $inf$ (right censoring).

%%%%%%%%%%%%%%%%%%%%%%%%%%%%%%%%%%%%%%%%%%%% LIKELIHOOD %%%%%%%%%%%%%%%%%%%%%%%%%%%%%%%%%%
\subsection*{Likelihood}
For calculating likelihood, in the observations with no censoring, the pdf is used, and in censored observations, the cdf is used. Hence, the likelihood is given as:

\begin{equation}
l = \left (\prod_{exact} f(e_i) / \sigma\right) \left (\prod_{right} 1-F(e_i)\right) \left (\prod_{left} F(e_i)\right) \left (\prod_{interval} F(e_i^u) - F(e_i^l)\right )
\end{equation}

%\begin{equation}
%L = \prod_{i=1}^n \zeta_i
%\end{equation}
%
%where,
%\begin{equation} \label{zeta}
%\zeta_i = \begin{cases}
%          \zeta_{i,1} = F(z_i) & \mbox{if: } -\infty \textbf{ = } \underline t_i \mbox{ , } \overline t_i<\infty \\
%	  \zeta_{i,2} = 1- F(z_i) & \mbox{if: } -\infty < \underline t_i \mbox{ , }\overline t_i\textbf{ = }\infty \\
%	  \zeta_{i,3} = F(z_i^u) - F(z_i^l) & \mbox{if: } -\infty < \underline t_i \ne \overline t_i<\infty \\
%	  \zeta_{i,4} = d(z_i) / \sigma & \mbox{if: } -\infty < \underline t_i \textbf{ = } \overline t_i<\infty \\
%\end{cases}
%\end{equation}

where, "exact", "left", "right", and "interval" refer to uncensored, left censored, right censored and interval censored observations respectively, and $F$ is the cdf of the distribution. $e_i^u$, and $e_i^l$ are upper and lower endpoints for interval censored data.

%$\eta_i = x_i^T \beta$ is the vector of linear predictors.
%Clearly, for left censored observations, $\underline t_i = -inf$, hence we can substitute $L_i = 0$ in $\zeta_3$ to get $\zeta_2$. 
%Simlarly, for right censored observations, $\overline t_i = inf$, hence we can substitute $R_i = inf$ in $\zeta_3$ to get $\zeta_1$. \\

Hence the log likelihood is given as:
\begin{equation} \label{lik0}
LL = \sum_{exact} g_1(e_i) - \log(\sigma) + \sum_{right} g_2(e_i) + \sum_{left} g_3(e_i) + \sum_{interval} g_4(e_i^l, e_i^u)
\end{equation}
where $g_1 = \log(f)$, $g_2 = \log(1-F)$, $g_3 = \log(F)$, $g_4(e_i^l, e_i^u) = \log(F(e_i^u) - F(e_i^l))$.

%\begin{equation} \label{lik0}
%LL = \sum_{i=1}^n \log \zeta_i = \sum_{i=1}^n \gamma_i %(\beta, \underline t_i, \overline t_i)
%\end{equation}
%
%where, $\gamma_i = \log \zeta_i$.

%%%%%%%%%%%%%%%%%%%%%%%%%%%%%%%%%%%%%%%%%%%% SCORE %%%%%%%%%%%%%%%%%%%%%%%%%%%%%%%%%%
\subsection*{Score and Hessian}

Derivatives of the LL with respect to the regression parameters are:

\begin{equation}
 \frac{\partial LL}{\partial \beta_j} = \sum_{i=1}^n \frac{\partial g}{\partial \eta_i}\frac{\partial \eta_i}{\partial \beta_j} = \sum_{i=1}^n x_{ij} \frac{\partial g}{\partial \eta_i}
\end{equation}

\begin{equation}
 \frac{\partial^2 LL}{\partial \beta_j \beta_k} = \sum_{i=1}^n x_{ij} x_{ik} \frac{\partial^2 g}{\partial \eta_i^2}
\end{equation}

where $\eta_i = x_i^T \beta$ is the vector of linear predictors.



%For now, we consider that $\underline t_i = \overline t_i$ is not possible. Hence we are left with the types of censoring. \\
%We consider the parital derivative  of $\gamma_{i,3}$ wrt to the parameters $\beta$. We can simply obtain derivatives for
%$\gamma_{i,1}$ and $\gamma_{i,2}$ by setting $L_i = 0$ and $R_i = inf$ respectively.\\

%We use the following:
%\begin{equation} \label{dF}
%\frac{\partial F(R_i)}{\partial \beta_j} =  -x_{ij} F(R_i)[1-F(R_i)]
%\end{equation}

%\begin{equation}
%\frac{\partial \gamma_{i,3}}{\partial \beta_j} = - x_{ij} \frac{ [F(R_i) (1-F(R_i)) - F(L_i)(1-F(L_i))] }{F(R_i) - F(L_i)} = x_{ij} [F(L_i) + F(R_i) - 1]
%\end{equation}

\vspace{5mm}

Define $\mu_i = \frac{\partial g}{\partial \eta_i}$, where g is one of $g_1$ to $g_4$ depending on type of censoring in the $i^{th}$ observation, and $\mu = [\mu_1, ... \mu_n]^T$. Then, partial derivative of log-likelihood is given as:

\begin{equation}
\frac{\partial l(\beta)}{\partial \beta_j} = \sum_{i=1}^n x_{ij} \mu_i
\end{equation}

Hence, the score (gradient of log likelihood) is given as:

\begin{equation}
S = \nabla_{\beta} LL(\beta) = X^T \mu = \sum_{i=1}^n \mu_i \overline x_i
\end{equation}

%%%%%%%%%%%%%%%%%%%%%%%%%%%%%%%%%%%%%%%%%%%% HESSIAN %%%%%%%%%%%%%%%%%%%%%%%%%%%%%%%%%%
%\subsection{Hessian}



%Hessian is the second derivative of loglikelihood:
%
%\begin{equation}
%H = \frac{ \partial {g(\beta)}^T}{\partial \beta} = \sum_{i=1}^n (\nabla_{\beta} \mu_i) \overline x_i^T
%\end{equation}
%
%Using \ref{dF}, we have:
%
%\begin{equation}
%\frac{ \partial \mu_i}{\partial \beta_j} = -x_{ij} [F(R_i) (1-F(R_i)) +  F(L_i) (1-F(L_i))]
%\end{equation}
%
%\begin{equation}
%\nabla_{\beta} \mu_i = -\overline x_i [F(R_i) (1-F(R_i)) + F(L_i) (1-F(L_i))]
%\end{equation}

The hessian can be written as:
\begin{equation} \label{w_i}
H = \sum_{i=1}^n  \overline x_i \overline x_i^T \frac{\partial^2 g}{\partial \eta_i^2}
 = \sum_{i=1}^n  \overline x_i \overline x_i^T w_i
\end{equation}

%where, $w_i$ is given as:
%\begin{equation} \label{w_i}
%\begin{split}
%w_i & = [F(R_i) (1-F(R_i)) + F(L_i) (1-F(L_i))] \\
%	& = - \mu_i [F(R_i) - F(L_i)]
%\end{split}
%\end{equation}

Define $W=diag(w_1, ... w_n)$.
\begin{equation}
H = X^T W X
\end{equation}

%Clearly, the hessian is negative definite. Thus, the loglikelihood is strictly convex, and a unique global
%maximum exists.

%%%%%%%%%%%%%%%%%%%%%%%%%%%%%%%%%%%%%%%%%%%% IRLS  %%%%%%%%%%%%%%%%%%%%%%%%%%%%%%%%%%
\subsection*{IRLS}
We use Newton's algorithm to find MLE for the AFT model, using negative loglikelihood (NLL). The Newton update is as follows:
\begin{equation}
\begin{split}
\beta & = \widetilde{\beta} - H^{-1}\widetilde{S} \\ 
 	  & = \widetilde{\beta} - (X^T \widetilde{W} X)^{-1} X^T \widetilde{\mu} \\
 	  & = (X^T \widetilde{W} X)^{-1} ((X^T \widetilde{W} X) \widetilde{\beta} -  X^T \widetilde{\mu} ) \\
 	  & = (X^T \widetilde{W} X)^{-1} X^T (\widetilde{W} X \widetilde{\beta} -  \widetilde{\mu} ) \\
 	  & = (X^T \widetilde{W} X)^{-1} X^T (\widetilde{W} X \widetilde{\beta} -  \widetilde{\mu} )
\end{split}
\end{equation}

where, define the working response $\widetilde{z} = X \widetilde{\beta} - \widetilde{W}^{-1} \widetilde{\mu}$.
Here, the tilde denotes that the respective values are evaluated using the parameters from the previous step.

Hence, at each step we are solving a weighted least squares problem, which is a minimizer of:

\begin{equation} \label{z_i}
\sum_{i=1}^n \widetilde{w_i} (\widetilde{z_i} - \overline x_i^T \beta)^2
\end{equation}
%
%We can rewrite $z$ as:
%\begin{equation} \label{z_i}
%\begin{split}
%\widetilde{ z_i} & = x_i^T \widetilde{ \beta} + \frac{ \widetilde{ \mu_i}} {\widetilde{ w_i}} \\
%				 & = x_i^T \widetilde{ \beta} - \frac{ 1}{F(R_i) - F(L_i)}
%\end{split}
%\end{equation}
%
This algorithm is the iteratively reweighted least squares (IRLS), since at each iteration we solve a weighted least squares problem.

%%%%%%%%%%%%%%%%%%%%%%%%%%%%%%%%%%%%%%%%%%%% ELASTIC  %%%%%%%%%%%%%%%%%%%%%%%%%%%%%%%%%%
\subsection*{Elastic net penalty and coordinate descent}
We define the elastic net (L1 + L2) penalty as follows:
\begin{equation} \label{objective}
\lambda P_{\alpha}(\beta) = \lambda(\alpha \|\beta\|_1 + 1/2 (1-\alpha) \|\beta\|_2^2)
\end{equation}

Adding the elastic net (L1 + L2) penalty, we get the following penalized weighted least squares objective:
\begin{equation} \label{objective}
M = \sum_{i=1}^n \widetilde{w_i} (\widetilde{z_i} - \overline x_i^T \beta)^2
	 + \lambda P_{\alpha}(\beta)
\end{equation}

The subderivative of the optimization objective is given as:
\begin{equation}
\frac{ \partial M}{\partial \beta_k} = \sum_{i=1}^n \widetilde{w_i} x_{ik} (\widetilde{ z_i} - \overline x_i^T \beta ) + \lambda \alpha \mbox{ sgn}(\beta_k) + \lambda (1-\alpha)\beta_k
\end{equation}

where, sgn$(\beta_k)$ is 1 if $\beta_k > 1$, -1 if $\beta_k<0$ and 0 if $\beta_k = 0$.
%\begin{equation} \label{subderivative}
%\frac{\partial{f}}{\partial{w_j}} = \begin{cases}
%										\{-\frac{\partial{l(\beta)}}{\partial{w_j}} + \lambda_2 w_j -\lambda_1\} & \mbox{if }  w_j<0 \\
%										[-\frac{\partial{l(\beta)}}{\partial{w_j}} -\lambda_1, -\frac{\partial{l(\beta)}}{\partial{w_j}} +\lambda_1] & \mbox{if }  w_j=0 \\
%										\{-\frac{\partial{l(\beta)}}{\partial{w_j}} + \lambda_2 w_j +\lambda_1\} & \mbox{if } w_j>0 \\
%									\end{cases}
%\end{equation}

Using the subderivative, three cases of solutions for $\beta_k$ may be obtained. The solution is given by:

\begin{equation} \label{beta}
\hat \beta_k = \frac{S\left(\sum_{i=1}^n \widetilde{w_i} x_{ik} \left[\widetilde{ z_i} - \sum_{j \ne k} x_{ij} \beta_j \right], \lambda \alpha \right)}
					{\sum_{i=1}^p \widetilde{w_i} x_{ik}^2 + \lambda (1- \alpha)}
\end{equation}

where, S is the soft thresholding operator, and $w_i$ and $z_i$ are given in \ref{w_i} and \ref{z_i} respectively.

The coordinate descent algorithm works by
cycling through each $\beta_j$ in turn, keeping the others constant, and using the above estimate to calculate the optimal value
$\hat \beta_j$. This is repeated until convergence.

%%%%%%%%%%%%%%%%%%%%%%%%%%%%%%%%%%%%%%%%%%%% ALGORITHM  %%%%%%%%%%%%%%%%%%%%%%%%%%%%%%%%%%
\subsection*{Pathwise solution}
This section is borrowed from section 2.3 of \cite{a3}.
The iregnet function will return solutions for an entire path of vaules of $\lambda$, for a fixed $\alpha$.
We begin with $\lambda$ sufficiently large to set the solution $\beta = 0$, and decrease $\lambda$ until we arrive
near the unregularized solution. The solutions for each value of $\lambda$ are used as the initial
estimates of $\beta$ for the next $\lambda$ value. This is known as warm starting, and makes the algorithm efficient and stable.
To choose initial value of $\lambda$, we use Equation \ref{beta}, and notice that for $\frac{1}{n} \sum_{i=1}^n w_i(0) x_{ij} z(0)_i < \alpha \lambda$ for all $j$, then $\beta = 0$ minimizes the objective \ref{objective}. Thus,

\begin{equation} \label{lambda}
\lambda_{max} = max_j \frac{1}{n \alpha} \sum_{i=1}^n w_i(0) x_{ij} z(0)_i
\end{equation}

We will set $\lambda_{min} = \epsilon \lambda_{max}$ , and compute solutions over a grid of $m$ values, where $\lambda_j = \lambda_{max}(\lambda_{min} / \lambda_{max})^{j/m}$ for $j = 0, .., m$.

\subsection*{Algorithm}
The algorithm to be followed for fitting the distribution is:

\begin{algorithm}[H]
\SetAlgoLined
 Calculate $\lambda_{max}$ using equation \ref{lambda} \;
 \ForEach{$\lambda_j$ in $j=0, ..., m$}{
   Initialise $\widetilde{\beta}$ using solution $\hat \beta$ from previous $\lambda $\;
   \Repeat{convergence of $\hat \beta$}{
	Compute $\widetilde{ w_i}$ and $\widetilde{ z_i}$ \;
	Find $\hat \beta$ by solving the penalized weighted least square problem defined in equation \ref{objective} using coordinate descent \;
	Set $\widetilde{ \beta} = \hat \beta$ \;
   }
 }
 \caption{Overall optimization algorithm}
\end{algorithm}

\subsection*{Scale parameter}
So far, I have ignored the $\sigma$ parameter from the calculations and equations. This is only
reasonable if we treat $\sigma$ as fixed. However, in other cases, $\sigma$ needs to estimated along with the parameters $\beta$, by using the derivatives as listed below.

\subsection*{Derivatives}
Iterations are done with respect to $\log(\sigma)$ to prevent numerical underflow.
\begin{equation}
\begin{split}
\frac{\partial g_1}{\partial \eta} & = - \frac{1}{\sigma} \left [ \frac{f'(z)}{f(z)} \right ] \\
\frac{\partial g_4}{\partial \eta} & = - \frac{1}{\sigma} \left [ \frac{f(z^u) - f(z^l)} {F(z^u) - F(z^l)} \right ] \\
\frac{\partial^2 g_1}{\partial \eta^2} & = - \frac{1}{\sigma^2} \left [ \frac{f''(z)}{f(z)} \right ] - \left ({\partial g_1}/{\partial \eta} \right ) \\
\frac{\partial^2 g_4}{\partial \eta^2} & = - \frac{1}{\sigma^2} \left [ \frac{f'(z^u) - f'(z^l)} {F(z^u) - F(z^l)} \right ] - \left ({\partial g_4}/{\partial \eta} \right )^2 \\
\frac{\partial g_1}{\partial \log \sigma} & = - \left [ \frac{z f'(z)}{f(z)} \right ] \\
\frac{\partial g_4}{\partial \log \sigma} & = - \left [ \frac{z^u f(z^u) - z^l f(z^l)} {F(z^u) - F(z^l)} \right ] \\
\frac{\partial^2 g_1}{\partial (\log \sigma )^2} & = \left [ \frac{z^2 f''(z) + z f'(z)}{f(z)} \right ] - \left ({\partial g_1}/{\partial \log \sigma } \right )^2 \\
\frac{\partial^2 g_4}{\partial (\log \sigma )^2} & = \left [ \frac{(z^u )^2 f'(z^u) - (z^l )^2 f'(z^l)} {F(z^u) - F(z^l)} \right ] - (\partial g_1 / \partial \log \sigma) (1 + \partial g_1 / \partial \log \sigma ) \\
\frac{\partial^2 g_1}{\partial \eta \partial \log \sigma} & = \left [\frac{z f''(z)} {\sigma f(z)} \right ] - (\partial g_1 / \partial \eta) (1 + \partial g_1 / \partial \log \sigma ) \\
\frac{\partial^2 g_4}{\partial \eta \partial \log \sigma} & = \left [\frac{z^u f'(z^u ) - z^l f'(z^l)} {\sigma [F(z^u) - F(z^l)]} \right ] - (\partial g_4 / \partial \eta) (1 + \partial g_4 / \partial \log \sigma ) \\
\end{split}  		% END SPLIT
\end{equation}

Derivatives for $g_2$ can be obtained by setting $z_u$ to $\inf$ in the equations for $g_4$, and similarly for $g_3$.

The distribution specific values of $f(z)$, etc. are omitted.

\begin{thebibliography}{1}
\bibitem{a1} Survival - Terry M Therneau
\bibitem{a2} Machine Learning: A Probabilistic Perspective - Kevin Murphy
\bibitem{a3} Regularization Paths for Cox's Proportional Hazards Model via Coordinate Descent - Simon, Friedman, Hastie, Tibshirani
\bibitem{a4} AFT - TD Hocking
\end{thebibliography}
%\textbf{References:}
%\begin{itemize}
%\end{itemize}

\end{document}
